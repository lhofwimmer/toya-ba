\chapter{Tests}
\label{cha:tests}

Um die Funktionalität von \toya zu gewährleisten, sind Tests zu vollziehen. Diese Tests verlaufen in drei Schritte, wie folgt:
\begin{enumerate}
    \item Programmcode in \toya schreiben
    \item den kompilierten Bytecode analysieren
    \item Die Ausgabe des Programms überprüfen
\end{enumerate}

Die Analyse des Bytecode erfolgt mit dem, in der JDK inkludierten Werkzeug \texttt{javap}. Dieses erlaubt es, den Bytecode einer \texttt{class}-Datei in einer für Menschen leserlicher Form auszugeben. Der Aufruf erfolgt über die Kommandozeile im Format: \texttt{javap -c Main.class}. Das Argument \texttt{-c} zeigt zusätzlich die Bytecode Befehle innerhalb der Methoden an. Die Ausführung der Programme erfolgt über die Kommandozeile mit dem Befehl \texttt{java Main}.

Insgesamt gilt es, die einzelnen Sprachkonstrukte, wie zum Beispiel For-Schleifen und Variablendeklaration und anschließend umfangreichere Programme zu testen.

\section{Hello World}

\begin{ToyaCode}[numbers=none, caption={Quelltext des Hello World Programms}]
function main(args: string[]) {
    print("Hello World")
}
\end{ToyaCode}

\begin{JavaCode}[numbers=none, caption={Bytecode des Hello World Programms}]
public class Main {
    public static void main(java.lang.String[]);
        Code:
            0: getstatic     #12    // Field java/lang/System.out:Ljava/io/PrintStream;
            3: ldc           #14    // String Hello World
            5: invokevirtual #20    // Method java/io/PrintStream.println:(Ljava/lang/String;)V
            8: return
}
\end{JavaCode}

\begin{ToyaCode}[numbers=none, caption={Konsolen-Ausgabe des Hello World Programms}]
Hello World    
\end{ToyaCode}

\section{Funktionen}

\begin{ToyaCode}[numbers=none, caption={Funktionen}]
function title() {
    print("This is an addition:")
}

function add(lhs: int, rhs: int) -> int {
    lhs + rhs
}

function main(args: string[]) {
    title()
    print(add(1,2))
}
\end{ToyaCode}  

\begin{JavaCode}[numbers=none, caption={Bytecode für Funktionen}]
public class Main {
    public static void title();
        Code:
            0: getstatic     #12    // Field java/lang/System.out:Ljava/io/PrintStream;
            3: ldc           #14    // String This is an addition:
            5: invokevirtual #20    // Method java/io/PrintStream.println:(Ljava/lang/String;)V
            8: return

    public static int add(int, int);
        Code:
            0: iload_0
            1: iload_1
            2: iadd
            3: ireturn

    public static void main(java.lang.String[]);
        Code:
            0: invokestatic  #26    // Method title:()V
            3: getstatic     #12    // Field java/lang/System.out:Ljava/io/PrintStream;
            6: ldc           #27    // int 1
            8: ldc           #28    // int 2
            10: invokestatic  #30   // Method add:(II)I
            13: invokevirtual #33   // Method java/io/PrintStream.println:(I)V
            16: return
}    
\end{JavaCode}

\begin{ToyaCode}[numbers=none, caption={Konsolen-Ausgabe der Funktionen}]
This is an addition:
3  
\end{ToyaCode}


\section{Variablen}

\begin{ToyaCode}[numbers=none, caption={Variablen}]
function someFunction() -> int {
    return 8
}

function main(args: string[]) {
    var number = 123
    var word = "Hello World"
    var double = 123.456
    var bool = true
    var result = someFunction()

    print(number)
    print(word)
    print(double)
    print(bool)
    print(result)
}
\end{ToyaCode}

\begin{JavaCode}[numbers=none, caption={Bytecode von Variablen}]
public class Main {
    public static int someFunction();
        Code:
            0: ldc           #7                  // int 8
            2: ireturn
    
    public static void main(java.lang.String[]);
        Code:
            0: ldc           #10    // int 123
            2: istore_1
            3: ldc           #12    // String Hello World
            5: astore_2
            6: ldc2_w        #13    // double 123.456d
            9: dstore_3
            10: iconst_1
            11: istore        5
            13: invokestatic  #16   // Method someFunction:()I
            16: istore        6
            18: getstatic     #22   // Field java/lang/System.out:Ljava/io/PrintStream;
            21: iload_1
            22: invokevirtual #28   // Method java/io/PrintStream.println:(I)V
            25: getstatic     #22   // Field java/lang/System.out:Ljava/io/PrintStream;
            28: aload_2
            29: invokevirtual #31   // Method java/io/PrintStream.println:(Ljava/lang/String;)V
            32: getstatic     #22   // Field java/lang/System.out:Ljava/io/PrintStream;
            35: dload_3
            36: invokevirtual #34   // Method java/io/PrintStream.println:(D)V
            39: getstatic     #22   // Field java/lang/System.out:Ljava/io/PrintStream;
            42: iload         5
            44: invokevirtual #37   // Method java/io/PrintStream.println:(Z)V
            47: getstatic     #22   // Field java/lang/System.out:Ljava/io/PrintStream;
            50: iload         6
            52: invokevirtual #28   // Method java/io/PrintStream.println:(I)V
            55: return
}      
\end{JavaCode}

\begin{ToyaCode}[numbers=none, caption={Konsolen-Ausgabe der Variablen}]
123
Hello World
123.456
true
8    
\end{ToyaCode}

\section{Felder}

\begin{ToyaCode}[numbers=none, caption={Felder}]
function main(args: string[]) {
    var arr = new int[8]
    arr[3] = 15

    print(arr[3])
    print(arr[4])
}
\end{ToyaCode}

\begin{JavaCode}[numbers=none, caption={Bytecode von Felder}]
public class Main {
    public static void main(java.lang.String[]);
        Code:
             0: ldc           #7    // int 3
             2: ldc           #8    // int 4
             4: if_icmpgt     11
             7: iconst_0
             8: goto          12
            11: iconst_1
            12: ifne          20
            15: ldc           #9    // int 6
            17: goto          22
            20: ldc           #10   // int 5
            22: istore_1
            23: getstatic     #16   // Field java/lang/System.out:Ljava/io/PrintStream;
            26: iload_1
            27: invokevirtual #22   // Method java/io/PrintStream.println:(I)V
            30: ldc           #7    // int 3
            32: ldc           #8    // int 4
            34: if_icmplt     41
            37: iconst_0
            38: goto          42
            41: iconst_1
            42: ifne          56
            45: getstatic     #16   // Field java/lang/System.out:Ljava/io/PrintStream;
            48: ldc           #24   // String false branch
            50: invokevirtual #27   // Method java/io/PrintStream.println:(Ljava/lang/String;)V
            53: goto          64
            56: getstatic     #16   // Field java/lang/System.out:Ljava/io/PrintStream;
            59: ldc           #29   // String true branch
            61: invokevirtual #27   // Method java/io/PrintStream.println:(Ljava/lang/String;)V
            64: return
}       
\end{JavaCode}

\begin{ToyaCode}[numbers=none, caption={Konsolen-Ausgabe der Felder}]
15
0    
\end{ToyaCode}

\section{If-Verzweigungen}

\begin{ToyaCode}[numbers=none, caption={If-Verzweigungen}]
function main(args: string[]) {
    var value = if (3 > 4) 5 else 6
    print(value)

    if(3 < 4) {
        print("true branch")
    } else {
        print("false branch")
    }
}
\end{ToyaCode}

\begin{JavaCode}[numbers=none, caption={Byteocde der If-Verzweigungen}]
public class Main {
    public static void main(java.lang.String[]);
        Code:
             0: ldc           #7    // int 3
             2: ldc           #8    // int 4
             4: if_icmpgt     11
             7: iconst_0
             8: goto          12
            11: iconst_1
            12: ifne          20
            15: ldc           #9    // int 6
            17: goto          22
            20: ldc           #10   // int 5
            22: istore_1
            23: getstatic     #16   // Field java/lang/System.out:Ljava/io/PrintStream;
            26: iload_1
            27: invokevirtual #22   // Method java/io/PrintStream.println:(I)V
            30: ldc           #7    // int 3
            32: ldc           #8    // int 4
            34: if_icmplt     41
            37: iconst_0
            38: goto          42
            41: iconst_1
            42: ifne          56
            45: getstatic     #16   // Field java/lang/System.out:Ljava/io/PrintStream;
            48: ldc           #24   // String false branch
            50: invokevirtual #27   // Method java/io/PrintStream.println:(Ljava/lang/String;)V
            53: goto          64
            56: getstatic     #16   // Field java/lang/System.out:Ljava/io/PrintStream;
            59: ldc           #29   // String true branch
            61: invokevirtual #27   // Method java/io/PrintStream.println:(Ljava/lang/String;)V
            64: return
}
\end{JavaCode}

\begin{ToyaCode}[numbers=none, caption={Konsolen-Ausgabe der If-Verzweigungen}]
6
true branch    
\end{ToyaCode}

\section{For-Schleifen}

\begin{ToyaCode}[numbers=none, caption={For-Schleifen}]
function main(args: string[]) {
    var n = 200

    for (var i = 0; i <= n; i = i+10) {
        if (i == 20) {
            print("i is 20:")
        }
        print(i)
    }
}
\end{ToyaCode}

\begin{JavaCode}[numbers=none, caption={Bytecode der For-Schleife}]
public class Main {
    public static void main(java.lang.String[]);
        Code:
             0: ldc           #7    // int 200
             2: istore_1
             3: ldc           #8    // int 0
             5: istore_2
             6: iload_2
             7: iload_1
             8: if_icmple     15
            11: iconst_0
            12: goto          16
            15: iconst_1
            16: ifeq          59
            19: iload_2
            20: ldc           #9    // int 20
            22: if_icmpeq     29
            25: iconst_0
            26: goto          30
            29: iconst_1
            30: ifne          36
            33: goto          44
            36: getstatic     #15   // Field java/lang/System.out:Ljava/io/PrintStream;
            39: ldc           #17   // String i is 20:
            41: invokevirtual #23   // Method java/io/PrintStream.println:(Ljava/lang/String;)V
            44: getstatic     #15   // Field java/lang/System.out:Ljava/io/PrintStream;
            47: iload_2
            48: invokevirtual #26   // Method java/io/PrintStream.println:(I)V
            51: iload_2
            52: ldc           #27   // int 10
            54: iadd
            55: istore_2
            56: goto          6
            59: return
}
\end{JavaCode}

\begin{ToyaCode}[numbers=none, caption={Konsolen-Ausgabe der For-Schleife}]
0
10
i is 20:
20
30
40
50
60
70
80
90
100
110
120
130
140
150
160
170
180
190
200
\end{ToyaCode}

\section{Umfangreicheres Beispiel}
Die Tests bisher beschränkten sich auf einzelne Eigenschaften von \toya. Um aber auch das Zwischenspiel von mehreren Eigenschaften zu testen, wird nun noch ein umfangreicheres Programm analysiert. Dieses Programm ruft eine Funktion auf, welche Zeichenketten auf der Konsole ausgibt, definiert Variablen von Felder und primitiven Datentypen, iteriert über eine For-Schleife, welche eine If-Verzweigung enthält und verändert und liest Felder.

\begin{ToyaCode}[numbers=none, caption={Quelltext des umfangereicheren Beispiels}]
function printIntro(n: int) {
    print("Welcome to a simple toya program")
    print("Calculating numbers up to: ")
    print(n)
    print("---------")
}

function main(args: string[]) {
    var n = 200
    printIntro(n)

    for (var i = 0; i <= n; i = i+10) {
        if (i == 20) {
            print("i is 20:")
        }
        print(i)
    }

    var boolArr = new boolean[2]
    boolArr[1] = true
    print(boolArr[0])
    print(boolArr[1])
}
\end{ToyaCode}

Da das Anzeigen des Bytecodes bei diesem umfangereicheren Beispiel zu lange ist und alle Fragmente in den anderen Tests bereits zu sehen ist, wird der Bytecode an dieser Stelle ausgelassen

\begin{ToyaCode}[numbers=none, caption={Konsolen-Ausgabe des umfangereicheren Beispiels}]
Welcome to a simple toya program
Calculating numbers up to:
200
---------
0
10
i is 20:
20
30
40
50
60
70
80
90
100
110
120
130
140
150
160
170
180
190
200
false
true    
\end{ToyaCode}