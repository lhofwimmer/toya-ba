\chapter{Tests}
\label{cha:tests}

Um die Funktionalität von \toya zu gewährleisten, sind Tests zu vollziehen. Diese Tests verlaufen in drei Schritte, wie folgt:
\begin{enumerate}
    \item Programmcode in \toya schreiben
    \item den kompilierten Bytecode analysieren
    \item Die Ausgabe des Programms überprüfen
\end{enumerate}

Insgesamt gilt es, die einzelnen Sprachkonstrukte, wie zum Beispiel for-Schleifen und Variablendeklaration, und anschließend umfangreichere Programme zu testen.

\section{Hello World}

\begin{ToyaCode}[numbers=none, caption={Hello World}]
function main(args: string[]) {
    print("Hello World")
}
\end{ToyaCode}

\section{Funktionen}

\section{Variablen}

\section{Arrays}

\section{If-Verzweigungen}

\section{For-Schleifen}