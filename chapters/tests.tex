\chapter{Tests}
\label{cha:tests}

Um die Funktionalität von \toya zu gewährleisten, sind Tests zu vollziehen. Diese Tests verlaufen in drei Schritte, wie folgt:
\begin{enumerate}
    \item Programmcode in \toya schreiben
    \item Den kompilierten Bytecode analysieren
    \item Die Ausgabe des Programms überprüfen
\end{enumerate}

Die Analyse des Bytecode erfolgt mit dem, in der JDK inkludierten Werkzeug \texttt{javap}. Dieses erlaubt es, den Bytecode einer \texttt{class}-Datei in einer für Menschen leserlicher Form auszugeben. Der Aufruf erfolgt über die Kommandozeile im Format: \texttt{javap -c Main.class}. Das Argument \texttt{-c} zeigt zusätzlich die Bytecode Befehle innerhalb der Methoden an. Die Ausführung der Programme erfolgt über die Kommandozeile mit dem Befehl \texttt{java Main}.

Insgesamt gilt es, die einzelnen Sprachkonstrukte, wie zum Beispiel For-Schleifen und Variablendeklaration und anschließend umfangreichere Programme zu testen.

\section{Hello World}

Der erste Test stellt ein \textit{Hello World} Programm dar, wie in~\autoref{lst:test_1_source} zu sehen ist. Der Bytecode dieses Programms beschränkt sich auf einige wenige Befehle.~\autoref{lst:test_1_bytecode} zeigt den Bytecode. Konkret laden \texttt{getstatic} und \texttt{ldc} Referenzen zum \texttt{System.out} Objekt und der auszugebenden Zeichenkette. Anschließend ruft \texttt{invokevirtual} die \texttt{println} Funktion mit der Zeichenkette auf. Als Resultat gibt dieser Test die Zeichenkette \textit{Hello World} auf der Konsole aus, zu sehen in~\autoref{lst:test_1_output}.

\begin{ToyaCode}[numbers=none, caption={Quelltext des Hello World Programms},label=lst:test_1_source]
function main(args: string[]) {
    print("Hello World")
}
\end{ToyaCode}

\begin{JavaCode}[numbers=none, caption={Bytecode des Hello World Programms},label=lst:test_1_bytecode]
public class Main {
    public static void main(java.lang.String[]);
        Code:
            0: getstatic     #12    // Field java/lang/System.out:Ljava/io/PrintStream;
            3: ldc           #14    // String Hello World
            5: invokevirtual #20    // Method java/io/PrintStream.println:(Ljava/lang/String;)V
            8: return
}
\end{JavaCode}

\begin{ToyaCode}[numbers=none, caption={Konsolen-Ausgabe des Hello World Programms},label=lst:test_1_output]
Hello World    
\end{ToyaCode}

\section{Funktionen}

Der zweite Test beschäftigt sich mit der Verwendung von Funktionen und dem Arbeiten mit Funktionsergebnissen.~\autoref{lst:test_2_source} zeigt den Quelltext des Programms. Als erstes ruft dieser Test die \texttt{title} Funktion auf. Diese Funktion gibt die Zeichenkette \textit{This is an addition:} auf der Konsole aus. Anschließend ruft das Programm die \texttt{print} Funktion mit dem Ergebnis der \texttt{add} Funktion auf. Die \texttt{add} Funktion addiert in diesem Fall die ganzzahligen Wertliterale \texttt{1} und \texttt{2}. Dementsprechend gibt die \texttt{print} Funktion den Wert \texttt{3} auf der Konsole aus, wie in~\autoref{lst:test_2_output} zu sehen ist.

Der erste Test behandelt die Ausgabe via der \texttt{print} Funktion, daher wird nicht näher darauf eingegangen. Die \texttt{add} Funktion lädt mit den Opcodes \texttt{iload\_0} und \texttt{iload\_1} die beiden ganzzahligen Parameter der Funktion. \texttt{iadd} addiert anschließend die beiden Werte und liefert diese mit \texttt{ireturn} zurück. Der vollständige Bytecode für die drei Funktionen ist in~\autoref{lst:test_2_bytecode} zu sehen.

\begin{ToyaCode}[numbers=none, caption={Funktionen},label=lst:test_2_source]
function title() {
    print("This is an addition:")
}

function add(lhs: int, rhs: int) -> int {
    lhs + rhs
}

function main(args: string[]) {
    title()
    print(add(1,2))
}
\end{ToyaCode}  

\begin{JavaCode}[numbers=none, caption={Bytecode für Funktionen},label=lst:test_2_bytecode]
public class Main {
    public static void title();
        Code:
            0: getstatic     #12    // Field java/lang/System.out:Ljava/io/PrintStream;
            3: ldc           #14    // String This is an addition:
            5: invokevirtual #20    // Method java/io/PrintStream.println:(Ljava/lang/String;)V
            8: return

    public static int add(int, int);
        Code:
            0: iload_0
            1: iload_1
            2: iadd
            3: ireturn

    public static void main(java.lang.String[]);
        Code:
            0: invokestatic  #26    // Method title:()V
            3: getstatic     #12    // Field java/lang/System.out:Ljava/io/PrintStream;
            6: ldc           #27    // int 1
            8: ldc           #28    // int 2
            10: invokestatic  #30   // Method add:(II)I
            13: invokevirtual #33   // Method java/io/PrintStream.println:(I)V
            16: return
}    
\end{JavaCode}

\begin{ToyaCode}[numbers=none, caption={Konsolen-Ausgabe der Funktionen},label=lst:test_2_output]
This is an addition:
3  
\end{ToyaCode}

\section{Variablen}

Der dritte Test behandelt die Deklaration und Initialisierung von Variablen. Als erstes wird eine Variable pro Typ mit Wertliteralen und eine weitere Variable mit einem Funktionsergebnis angelegt, wie in~\autoref{lst:test_3_source} zu sehen ist. Anschließend wird jeder Wert auf der Konsole ausgegeben, siehe dazu~\autoref{lst:test_3_output}.

Die beiden Opcodes \texttt{<type>load} und \texttt{<type>store} laden und speichern respektive den Wert einer lokalen Variable. Der Bytecode ist unter~\autoref{lst:test_3_bytecode} zu sehen.

\begin{ToyaCode}[numbers=none, caption={Variablen},label=lst:test_3_source]
function someFunction() -> int {
    return 8
}

function main(args: string[]) {
    var number = 123
    var word = "Hello World"
    var double = 123.456
    var bool = true
    var result = someFunction()

    print(number)
    print(word)
    print(double)
    print(bool)
    print(result)
}
\end{ToyaCode}

\begin{JavaCode}[numbers=none, caption={Bytecode von Variablen},label=lst:test_3_bytecode]
public class Main {
    public static int someFunction();
        Code:
            0: ldc           #7                  // int 8
            2: ireturn
    
    public static void main(java.lang.String[]);
        Code:
            0: ldc           #10    // int 123
            2: istore_1
            3: ldc           #12    // String Hello World
            5: astore_2
            6: ldc2_w        #13    // double 123.456d
            9: dstore_3
            10: iconst_1
            11: istore        5
            13: invokestatic  #16   // Method someFunction:()I
            16: istore        6
            18: getstatic     #22   // Field java/lang/System.out:Ljava/io/PrintStream;
            21: iload_1
            22: invokevirtual #28   // Method java/io/PrintStream.println:(I)V
            25: getstatic     #22   // Field java/lang/System.out:Ljava/io/PrintStream;
            28: aload_2
            29: invokevirtual #31   // Method java/io/PrintStream.println:(Ljava/lang/String;)V
            32: getstatic     #22   // Field java/lang/System.out:Ljava/io/PrintStream;
            35: dload_3
            36: invokevirtual #34   // Method java/io/PrintStream.println:(D)V
            39: getstatic     #22   // Field java/lang/System.out:Ljava/io/PrintStream;
            42: iload         5
            44: invokevirtual #37   // Method java/io/PrintStream.println:(Z)V
            47: getstatic     #22   // Field java/lang/System.out:Ljava/io/PrintStream;
            50: iload         6
            52: invokevirtual #28   // Method java/io/PrintStream.println:(I)V
            55: return
}      
\end{JavaCode}

\begin{ToyaCode}[numbers=none, caption={Konsolen-Ausgabe der Variablen},label=lst:test_3_output]
123
Hello World
123.456
true
8    
\end{ToyaCode}

\section{Felder}

Der vierte Test behandelt die Deklaration, Initialisierung und den Zugriff auf Felder. Der Quelltext ist in~\autoref{lst:test_4_source} zu sehen. Das Programm legt an erster Stelle ein acht Elemente großes Feld vom Typ \texttt{int} an. Dem Index \texttt{3} wird das Wertliteral \texttt{15} zugewiesen. Anschließend gibt das Programm die Indizes \texttt{3} und \texttt{4} auf der Konsole aus. Wie erwartet liegt in \texttt{arr[3]} der Wert \texttt{15} und in \texttt{arr[4]} der Wert \texttt{0}, wie an der Konsolenausgabe in~\autoref{lst:test_4_output} zu sehen ist. Primitive Werte haben in Java immer einen Standardwert, was bei Ganzzahlen \texttt{0} ist. Obwohl \texttt{arr[4]} kein Wert zugewiesen wurde, ist daher trotzdem \texttt{0} auf der Konsole zu sehen.

Der Opcode \texttt{newarray} legt ein neues Feld an. Mit den Opcodes \texttt{iaload} und \texttt{iastore} ist der Zugriff auf Elemente des Feldes möglich. Der Bytecode ist in~\autoref{lst:test_4_bytecode} zu sehen.

\begin{ToyaCode}[numbers=none, caption={Felder},label=lst:test_4_source]
function main(args: string[]) {
    var arr = new int[8]
    arr[3] = 15

    print(arr[3])
    print(arr[4])
}
\end{ToyaCode}

\begin{JavaCode}[numbers=none, caption={Bytecode von Felder},label=lst:test_4_bytecode]
public class Main {
    public static void main(java.lang.String[]);
        Code:
             0: ldc           #7    // int 8
             2: newarray       int
             4: astore_1
             5: aload_1
             6: ldc           #8    // int 3
             8: ldc           #9    // int 15
            10: iastore
            11: getstatic     #15   // Field java/lang/System.out:Ljava/io/PrintStream;
            14: aload_1
            15: ldc           #8    // int 3
            17: iaload
            18: invokevirtual #21   // Method java/io/PrintStream.println:(I)V
            21: getstatic     #15   // Field java/lang/System.out:Ljava/io/PrintStream;
            24: aload_1
            25: ldc           #22   // int 4
            27: iaload
            28: invokevirtual #21   // Method java/io/PrintStream.println:(I)V
            31: return
}           
\end{JavaCode}

\begin{ToyaCode}[numbers=none, caption={Konsolen-Ausgabe der Felder},label=lst:test_4_output]
15
0    
\end{ToyaCode}

\section{If-Verzweigungen}

Der fünfte Test behandelt die Verwendung von If-Verzweigungen als Ausdruck und Anweisung. Der erste Teil des Quelltextes verwendet die If-Verzweigung als Ausdruck um einer Variable einen Wert zuzuweisen und diese auf der Konsole auszugeben. Der zweite Teil verwendet die If-Verzweigung als Anweisung um eine Zeichenkette je nach Zweig auszugeben. Der Quelltext ist in~\autoref{lst:test_5_source} zu sehen.~\autoref{lst:test_5_output} zeigt die Konsolenausgabe.

Im Bytecode sind die \texttt{goto} Opcodes zu beachten. Mithilfe \texttt{goto} ist der Sprung zwischen Anweisungen im Bytecode möglich. Jede Anweisung ist mit einer Nummer versehen. Diese Nummer ist bei \texttt{goto} anzugeben, um den Zielort des Sprungs zu bestimmen. Der Bytecode ist unter~\autoref{lst:test_5_bytecode} zu finden.

\begin{ToyaCode}[numbers=none, caption={If-Verzweigungen},label=lst:test_5_source]
function main(args: string[]) {
    var value = if (3 > 4) 5 else 6
    print(value)

    if(3 < 4) {
        print("true branch")
    } else {
        print("false branch")
    }
}
\end{ToyaCode}

\begin{JavaCode}[numbers=none, caption={Bytecode der If-Verzweigungen},label=lst:test_5_bytecode]
public class Main {
    public static void main(java.lang.String[]);
        Code:
             0: ldc           #7    // int 3
             2: ldc           #8    // int 4
             4: if_icmpgt     11
             7: iconst_0
             8: goto          12
            11: iconst_1
            12: ifne          20
            15: ldc           #9    // int 6
            17: goto          22
            20: ldc           #10   // int 5
            22: istore_1
            23: getstatic     #16   // Field java/lang/System.out:Ljava/io/PrintStream;
            26: iload_1
            27: invokevirtual #22   // Method java/io/PrintStream.println:(I)V
            30: ldc           #7    // int 3
            32: ldc           #8    // int 4
            34: if_icmplt     41
            37: iconst_0
            38: goto          42
            41: iconst_1
            42: ifne          56
            45: getstatic     #16   // Field java/lang/System.out:Ljava/io/PrintStream;
            48: ldc           #24   // String false branch
            50: invokevirtual #27   // Method java/io/PrintStream.println:(Ljava/lang/String;)V
            53: goto          64
            56: getstatic     #16   // Field java/lang/System.out:Ljava/io/PrintStream;
            59: ldc           #29   // String true branch
            61: invokevirtual #27   // Method java/io/PrintStream.println:(Ljava/lang/String;)V
            64: return
}
\end{JavaCode}

\begin{ToyaCode}[numbers=none, caption={Konsolen-Ausgabe der If-Verzweigungen},label=lst:test_5_output]
6
true branch    
\end{ToyaCode}

\section{For-Schleifen}

Der sechste Test behandelt For-Schleifen. Der Schleifenkopf initialisert und deklariert eine Zählvariable \texttt{i} mit dem Wert \texttt{0}. Nach jedem Schleifendurchlauf prüft die Abbruchbedingung, ob \texttt{i} den Wert der zuvor angelegten Variable \texttt{n} überschreitet. \texttt{n} wurde mit dem Wert \texttt{200} initialisert. Der Wert von \texttt{i} erhöht sich nach jedem Schleifendurchlauf um zehn. Im Schleifenkörper befindet sich eine \texttt{print}-Anweisung, die den Wert der Zählvariable auf der Konsole ausgibt. Der Quelltext ist in~\autoref{lst:test_6_source} zu sehen. Neben If-Verzweigungen sind \texttt{goto} Anweisungen auch bei For-Schleifen in Verwendung. Nach jedem Schleifendurchlauf springt das Programm zur Abbruchbedingung zurück. Evaluiert die Abbruchbedingung zu \texttt{true}, springt das Programm zum \texttt{return} und beendet die Ausführung. Der Bytecode ist unter~\autoref{lst:test_6_bytecode} zu sehen. Auf der Konsole sind die Werte von 0 bis 200 in zehner Inkrementen zu sehen. Die Ausgabe ist in~\autoref{lst:test_6_output} zu finden.

\begin{ToyaCode}[numbers=none, caption={For-Schleifen},label=lst:test_6_source]
function main(args: string[]) {
    var n = 200

    for (var i = 0; i <= n; i = i+10) {
        print(i)
    }
}
\end{ToyaCode}

\begin{JavaCode}[numbers=none, caption={Bytecode der For-Schleife},label=lst:test_6_bytecode]
public class Main {
    public static void main(java.lang.String[]);
        Code:
             0: ldc           #7    // int 200
             2: istore_1
             3: ldc           #8    // int 0
             5: istore_2
             6: iload_2
             7: iload_1
             8: if_icmple     15
            11: iconst_0
            12: goto          16
            15: iconst_1
            16: ifeq          34
            19: getstatic     #14   // Field java/lang/System.out:Ljava/io/PrintStream;
            22: iload_2
            23: invokevirtual #20   // Method java/io/PrintStream.println:(I)V
            26: iload_2
            27: ldc           #21   // int 10
            29: iadd
            30: istore_2
            31: goto          6
            34: return
}      
\end{JavaCode}

\begin{ToyaCode}[numbers=none, caption={Konsolen-Ausgabe der For-Schleife},label=lst:test_6_output]
0
10
20
30
40
50
60
70
80
90
100
110
120
130
140
150
160
170
180
190
200
\end{ToyaCode}

\section{Umfangreicheres Beispiel}
Die Tests bisher beschränkten sich auf einzelne Eigenschaften von \toya. Um aber auch das Zusammenspiel von mehreren Eigenschaften zu testen, wird nun noch ein umfangreicheres Programm getestet. Dieses Programm ruft eine Funktion auf, welche Zeichenketten auf der Konsole ausgibt, definiert Variablen von Felder und primitiven Datentypen, iteriert über eine For-Schleife, welche eine If-Verzweigung enthält und verändert und liest Felder. Der Quelltext ist in~\autoref{lst:test_7_source} zu sehen.

\begin{ToyaCode}[numbers=none, caption={Quelltext des umfangereicheren Beispiels},label=lst:test_7_source]
function printIntro(n: int) {
    print("Welcome to a simple toya program")
    print("Calculating numbers up to: ")
    print(n)
    print("---------")
}

function main(args: string[]) {
    var n = 200
    printIntro(n)

    for (var i = 0; i <= n; i = i+10) {
        if (i == 20) {
            print("i is 20:")
        }
        print(i)
    }

    var boolArr = new boolean[2]
    boolArr[1] = true
    print(boolArr[0])
    print(boolArr[1])
}
\end{ToyaCode}

Da das Anzeigen des Bytecodes bei diesem umfangereicheren Beispiel zu lange ist und alle Fragmente in den anderen Tests bereits zu sehen sind, wird der Bytecode an dieser Stelle ausgelassen.~\autoref{lst:test_7_output} zeigt die Konsolenausgabe des siebten Tests.

\begin{ToyaCode}[numbers=none, caption={Konsolen-Ausgabe des umfangereicheren Beispiels},label=lst:test_7_output]
Welcome to a simple toya program
Calculating numbers up to:
200
---------
0
10
i is 20:
20
30
40
50
60
70
80
90
100
110
120
130
140
150
160
170
180
190
200
false
true    
\end{ToyaCode}