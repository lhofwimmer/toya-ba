\chapter{Generierung des Syntaxbaums}
\label{cha:antlr}

ANTLR (\textbf{AN}other \textbf{T}ool for \textbf{L}anguage \textbf{R}ecognition) ist ein seit 1989 von Terrence Parr entwickeltes Werkzeug zum Erzeugen von syntaktischen und lexikalischen Analysatoren und Compilern. Die Benutzer:in definiert diese mit einer Grammatik und erzeugt daraus dann mithilfe eines von ANTLR zur Verfügung gestelltem Kommandozeilen-Werkzeug einen lexikalischen und syntaktischen Analysator in der gewünschten Ziel-Sprache. ANTLR unterstützt unter anderem Java, C\#, Python, JavaScript, Go, C++, Swift, PHP und Dart \parencite{antlrtargets}. ANTLR bietet als aktuellste Version 4 an, welche große Unterschiede - allen voran der Umstieg auf eine effizientere Analyse-Methodik - zu Version 3 bietet. Der Artikel \textcite{antlrMegaTutorial} dient als Grundlage für dieses Kapitel.

ANTLR findet auch im professionellen Umfeld Verwendung. So verwendet Twitter ANTLR zur Syntax-Analyse von über 2 Milliarden Suchanfragen pro Tag \parencite{antlrWebsite}. Hadoop verwendet ANTLR zur Syntax-Analyse von \textit{Hive} und \textit{Pig} und Netbeans analysiert den Syntax von \textit{C++} mithilfe ANTLR \parencite{antlrabout}.

Adaptive LL(*) Syntax-Analyse stellt den größten Unterschied zwischen Version drei und vier von ANTLR dar. ALL(*) ist ein neuer von Terrence Parr entwickelter Syntax-Analyse-Ansatz, welcher theoretisch zwar eine Laufzeitkomplexität von $\mathcal{O}(n^4)$ hat, praktisch gesehen aber lineare Performanz aufweist. Im Gegensatz zur traditionellen LL- oder LR-Syntax-Analyse, analysiert ALL(*) die gesamte Eingabe, um das Syntax-Analyse-Entscheidungsdiagramm zu konstruieren, das die Analyse verwendet. ALL(*) verwendet eine Technik namens adaptive Vorwärtsanalyse, um den Syntax-Analyse-Entscheidungsbaum zu verbessern. Diese Technik kombiniert Vorwärts- und Rückwärtsanalyse, um die Vorhersage der nächsten \textit{Token} zu verbessern \parencite{parr2014adaptive}.

Der große Vorteil von ANTLR gegenüber selbst entwickelten Lösungen zum Erzeugen von Syntaxbäumen ist die Effizienz mit welcher neue Grammatikregeln definiert werden können. Diese Effizienz und Leichtigkeit in der Umsetzung hat jedoch auch seine Kosten. Da ANTLR einen umfangreichen Syntaxbaum erzeugt und es sehr unwahrscheinlich ist, dass das Programm alle Daten des Syntaxbaums benötigt, kommt es zu einem Mehraufwand ohne Nutzen.

Deswegen verwenden alle größeren Programmiersprachen (C++, Python, C\#, Java, etc.) selbst entwickelte syntaktische und lexikalische Analysatoren um die Übersetzungszeit substantiell zu reduzieren.

Da \toya über eine experimentelle Programmiersprache nicht hinaus geht und es den programmatischen Aufwand übersteigt, wurde gegen eine maßgeschneiderte Lösung für \toya entschieden. Die Übersetzungszeiten unter Verwendung von ANTLR sind im Rahmen von \toya akzeptabel. Gegen die Implementierung der Analyse durch reguläre Ausdrücke sprechen mehrere Gründe:
\begin{itemize}
    \item Keine rekursive Syntax-Analyse möglich.
    \item Programmelemente, die an allen Stellen im Code (Kommentare zum Beispiel) auftauchen können, sind an allen potenziellen Stellen im Regex-Ausdruck zu berücksichtigen. Dies führt zu Redundanz.
    \item Reguläre Ausdrücke wachsen schnell und sind schwer zu verwalten. Da Programmiersprachen typischerweise auch in ihrem Funktionsumfang wachsen, sind reguläre Ausdrücke nicht dafür geeignet und führen zu einer schlechten Skalierbarkeit.
\end{itemize}

% Welche Sprachen unterstützt https://github.com/antlr/antlr4/blob/master/doc/targets.md
% Differences Antlr 3 und 4: https://github.com/antlr/antlr4/blob/master/doc/faq/general.md

\section{Grammatik-Definition}

Die Grammatik-Definition in einer \texttt{g4}-Datei ist der Ausgangspunkt für alle weiteren Schritte in ANTLR. Diese Datei beinhaltet alle Regeln für die lexikalischen und syntaktische Analyse und Meta Informationen anhand welcher Eingabe-Dateien abzuarbeiten sind. Meta Informationen befinden sich typischerweise am Beginn der Datei.

Da Leerzeichen in der Regel unwichtig sind und und keine Relevanz für die Semantik der Sprachen haben (abgesehen von Ausnahmefällen wie Python), ignoriert man diese. Dies erfolgt mithilfe der Anweisung \texttt{\string[ \textbackslash t\textbackslash n\textbackslash r\string]+ -> skip}, welche angibt, dass Leerzeichen bei der Abarbeitung einer Eingabedatei zu überspringen sind.
%  Diese Syntax findet auch Einsatz bei \textit{Channels}. Anstatt, wie im Falle von Leerzeichen, die Zeichen wegzuwerfen, 

Ob eine Regel die lexikalische oder syntaktische Analyse betrifft, hängt vom Anfangsbuchstaben dieser Regel ab. Ist das erste Zeichen ein Großbuchstabe, betrifft es die lexikalische Analyse; wenn nicht, die syntaktische Analyse. Typischerweise werden als Erstes Regeln für die syntaktische Analyse und als Zweites Regeln für den die lexikalische Analyse definiert. Die Reihenfolge der lexikalischen Analysator-Regeln ist von Relevanz, da in derselben Reihenfolge ANTLR diese Regeln analyisiert.~\autoref{lst:antlr_example} zeigt den Aufbau einer exemplaren g4-Datei.

\begin{AntlrCode}[numbers=none, caption={Beispielhafter Aufbau einer Grammatik-Definition für Additionen}, label=lst:antlr_example]
grammar: addition;

// parser rules
operation  : NUMBER '+' NUMBER ;

// lexer rules
NUMBER     : [0-9]+ ;
WHITESPACE : [ \t\n\r]+ -> skip ;
\end{AntlrCode}

\section{Listener}

Um die Ergebnisse des Syntaxbaums abarbeiten zu können, bietet ANTLR zwei Entwurfsmuster an: \visitor und \listener. Die Implementierung dieser Entwurfsmuster erzeugt die Benutzer:in mithilfe des Kommandozeilen-Werkzeug \texttt{antlr4} anhand der anzugebenden Grammatik-Datei. Außerdem gibt die Benutzer:in zusätzlich noch an, ob entweder die Implementierung für \visitor oder \listener oder für Beide zu generieren sind. Sollen keine \listener generiert werden, ist dies via dem Parameter \texttt{-no-listener} anzugeben.

Ein \listener hat im Gegensatz zu einem \visitor keinen Einfluss auf den Analyse-Vorgang. Stattdessen ruft der \textit{Tree Walker}, der den Syntaxbaum traversiert, die von ANTLR generierten Methoden für den richtigen Knotentyp anhand der Analyse-Regeln auf. Diese Methoden des \listeners liefern keinen Wert zurück, was die Verwaltung eines abstrakten Syntaxbaums erschwert. Aufgrund der Komplexität von \toya sind \listener daher nicht empfehlenswert.

\listener bieten sich gut für Zusatzverhalten an, welches den Syntaxbaum nicht verändert. Typische Verwendungszwecke für \listener sind das Protokollieren von Informationen oder Ermitteln von Metadaten.

\section{Visitor}

\visitor ist ein Entwurfsmuster, welches das Ausführen eine Operation auf den Elementen einer Objektstruktur ermöglicht. Die Klassen dieser Elemente oder die Struktur selbst wird dabei nicht verändert. Das Entwurfsmuster besteht aus zwei grundlegenden Komponenten: den Elementen der Objektstruktur und dem \visitor, der die Operation auf den Elementen ausführt. Die Elemente der Struktur implementieren eine Schnittstelle mit einer Funktion, um einen \visitor auf das Element anzuwenden. Der \visitor selbst definiert Methoden für jede Klasse von Elementen, die er besuchen kann.

Um das Entwurfsmuster zu nutzen, ruft man die \texttt{accept} Methode des \visitors auf dem Wurzelelement der Struktur auf. Diese Methode wiederum ruft die entsprechende Methode im \visitor auf, wodurch der \visitor das Element besucht. Das Element gibt sich selbst als Parameter an den \visitor weiter, wodurch dieser auf die Eigenschaften und Methoden des Elements zugreifen und eine Operation darauf ausführen kann. Ein mögliches Problem hierbei ist, dass Fehler leichter enstehen können. Vergisst die Entwickler:in auf den Aufruf einer notwendigen \texttt{accept} Methode, kommt es nicht zum Wechsel in einen Ausnahmezustand. Stattdessen ignoriert die syntaktische Analyse die \textit{Token}.

Das \visitor Entwurfsmuster hat den Vorteil, dass es das Open-Closed-Prinzip unterstützt, da neue Operationen durch die Erstellung neuer \visitor-Klassen hinzugefügt werden können, ohne die existierenden Elementklassen zu ändern. Außerdem können komplexe Operationen auf der Objektstruktur durchgeführt werden, indem man mehrere \visitor-Klassen erstellt, die jeweils eine Teiloperation durchführen.~\autoref{lst:antlr_visitor} zeigt die Implementierung eines \visitors, der Wertliterale verarbeitet.

\begin{KotlinCode}[numbers=none, caption={Implementierung eines \visitors für Wertliterale.}, label=lst:antlr_visitor]
class ExpressionVisitor(val scope: Scope) : toyaBaseVisitor<Expression>() {
    override fun visitValue(valueContext: toyaParser.ValueContext): Expression {
        val value = valueContext.text
        val type = TypeResolver.getFromValue(value)
        return Value(value, type)
    }
    // Rest of class...
}
\end{KotlinCode}