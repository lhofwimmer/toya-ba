\chapter{Zusammenfassung, Schlüsse und Lehren}
\label{cha:Schluss}

% Über Testbarkeit reden -> programmfragmente als Unit-Tests implementieren. Also z.b. source code von array def als unit-test etc. -> ermöglicht leichtes Testen.

% Im großen ganzen ist \toya in vielen Aspekten eine suboptimale Lösung, die in einer \textit{Version 2.0} vieles anders umgesetzt haben würde, aber sie funktioniert.

% ANTLR sehr nice, auch wenn die nachteile klar sind. grammar definition super. visitor pattern sehr gut verwendbar. generell ANTLR sehr gutes API design. would reuse this library
% in zukunft byte buddy statt asm? mehr abstraktion von vorteil
% sehr lehrreich tho