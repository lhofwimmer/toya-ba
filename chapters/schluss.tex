\chapter{Zusammenfassung, Schlüsse und Lehren}
\label{cha:Schluss}

% Über Testbarkeit reden -> programmfragmente als Unit-Tests implementieren. Also z.b. source code von array def als unit-test etc. -> ermöglicht leichtes Testen.

% Im großen ganzen ist \toya in vielen Aspekten eine suboptimale Lösung, die in einer \textit{Version 2.0} vieles anders umgesetzt haben würde, aber sie funktioniert.

% ANTLR sehr nice, auch wenn die nachteile klar sind. grammar definition super. visitor pattern sehr gut verwendbar. generell ANTLR sehr gutes API design. would reuse this library
% in zukunft byte buddy statt asm? mehr abstraktion von vorteil
% sehr lehrreich tho
% bei weiterentwicklung parser selbst schreiben.
% Einblick in Bytecode gibt mehr Verständniss hinsichtlich Abstraktion von Java

Ziel dieser Bachelorarbeit war es, sich mit der Implementierung einer eigens konzipierten Programmiersprache auseinanderzusetzen. Dieser Aufgabenstellung konnte erfolgreich durch die Implementierung von \toya und dessen Compiler nachgegangen werden. \Toya bietet in seiner finalen Form die Definition von Funktionen, Variablen, Kontrollflüssen und arithmetischen Ausdrücken. \Toya ist statisch typisiert und bietet als Datentypen Ganz- und Gleitkommazahlen, Zeichenketten und boole'sche Werte zur Initialisierung von Variablen, Funktionsparametern und Rückgabewerten.

Positiv hervorzuheben ist die Arbeit mit ANTLR. Nicht aufgrund der Effizienz, sondern aufgrund der ausgezeichneten Eingliederung in den Arbeitsablauf stellt es sich als besonders gutes Entwicklungswerkzeug heraus. Mithilfe eines Intellij IDEA Plugins können Textfragmente hinsichtlich ihrer Validität für eine spezifizierte Grammatik verifiziert und mögliche Fehler umgehend erkannt und behoben werden. Die Grammatikdefinition anhand g4-Dateien ermöglicht eine klare Übersicht darüber, wie die Programmiersprache nun konkret strukturiert ist.

\Toya und dessen Komponenten können auch leicht getestet werden. Als Eingabewert dienen Sprachfragmente, wie zum Beispiel eine Variablendeklaration. Als Ausgabewert, welcher anschließend auf Validität des Ergebnisses zu vergleichen ist, kann der generierte Bytecode verwendet werden. Durch diese textuelle Ein- und Ausgabe wird \toya zur leicht testbaren Sprache. Wächst nun der Umfang von \toya, so steigt der Testaufwand nicht überproportional, sondern beschränkt sich auf die neuen Sprachaspekte.

Sollte Interesse an der Weiterentwicklung von \toya bestehen, wäre es sinnvoll, ANTLR durch einen eigens implementierten Analysator zu ersetzen, da der Zeitaufwand des vollständigen Syntaxbaums mit steigender Komplexität erheblich zunimmt. Ein eigens implementierter Analysator kann viele Teile des Syntaxbaumes, die ANTLR generieren würde, komplett ignorieren und zum Beispiel im ersten Schritt der syntaktischen Analyse überprüfen, ob Wertliterale zum Beispiel einen validen Ausdruck darstellen. Für weniger umfangreiche Grammatiken erweist sich ANTLR als sinnvoll und ist daher ohne Bedenken in diesem Kontext weiterzuempfehlen.
