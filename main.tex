%%% Dateikodierung: UTF-8
%%% äöüÄÖÜß  <-- keine deutschen Umlaute hier? UTF-faehigen Editor verwenden!

%%% Magic Comments zum Setzen der korrekten Parameter in kompatiblen IDEs
% !TeX encoding = utf8
% !TeX program = pdflatex 
% !TeX spellcheck = de_DE
% !BIB program = biber

\documentclass[bachelor,german,smartquotes]{hgbthesis}
% Zulässige Optionen in [..]: 
%    Typ der Arbeit: 'diploma', 'master' (default), 'bachelor', 'internship' 
%    Hauptsprache: 'german' (default), 'english'
%    Option zur Umwandlung in typografische Anführungszeichen: 'smartquotes'
%    APA Zitierstil: 'apa'
%%%-----------------------------------------------------------------------------

\RequirePackage[utf8]{inputenc} % bei Verw. von lualatex oder xelatex entfernen!

\graphicspath{{images/}}  % Verzeichnis mit Bildern und Grafiken
\logofile{logo}           % Logo-Datei: images/logo.pdf (kein Logo: \logofile{})
\bibliography{references} % Biblatex-Literaturdatei (references.bib)

%%%-----------------------------------------------------------------------------
% Angaben für die Titelei (Titelseite, Erklärung etc.)
%%%-----------------------------------------------------------------------------

% \title{toya\\ Imperative Programmiersprache für die JVM}
% \author{Lukas Christian Hofwimmer}
% \programname{Universal Computing}

%\programtype{Fachhochschul-Bachelorstudiengang} % auswählen/editieren
% \programtype{Fachhochschul-Bachelorstudiengang}

% \placeofstudy{Hagenberg}
% \dateofsubmission{2021}{07}{15} % {YYYY}{MM}{DD}

% \advisor{FH-Prof. DI Dr. Heinz Dobler} % optional

%\strictlicense % restriktive Lizenz anstatt Creative Commons (nicht empfohlen!)

%%%-----------------------------------------------------------------------------
\begin{document}
%%%-----------------------------------------------------------------------------

%%%-----------------------------------------------------------------------------
\frontmatter                                       % Titelei (röm. Seitenzahlen)
%%%-----------------------------------------------------------------------------

% \maketitle
\includepdf[pages=1]{Titelblatt_Bachelorarbeit.pdf}
\includepdf[pages=2, pagecommand={\thispagestyle{empty}}]{Titelblatt_Bachelorarbeit.pdf}
\tableofcontents

\include{front/vorwort} % Ein Vorwort ist optional
\chapter{Kurzfassung}

\Toya ist eine stark typisierte, Turing-vollständige Programmiersprache für die \textit{Java Virtual Machine} (JVM). \Toya erlaubt die Erstellung von Funktionen, Variablen, arithmetischen Ausdrücken und Kontrollstrukturen in Form von If-Verzweigungen und For-Schleifen.

Variablen können vom Typ \texttt{int}, \texttt{double}, \texttt{string} oder \texttt{boolean} sein. Ebenso ist es möglich, Felder dieser Typen anzulegen. Die explizite Angabe des Variablentyps ist mit Ausnahme von Feldern nicht möglich. Stattdessen ermittelt \toya anhand des zugewiesenen Ausdrucks den Typ der Variable.

Der \toya Compiler ist in zwei Teile unterteilt: Frontend und Backend. Das Frontend kümmert sich um die Verarbeitung des Quelltextes um daraus einen abstrakten Syntaxbaum zu erzeugen. Das Backend erzeugt anschließend anhand des abstrakten Syntaxbaums Bytecode für die JVM. Als Produkt liefert \toya class-Dateien, die die JVM verarbeitet und auswertet. Der Compiler für \toya ist vollständig in Kotlin/JVM implementiert.

Das Frontend verwendet als zentrale Bibliothek ANTLR in der vierten Version. ANTLR ist ein Syntax-Analysator-Generator, der anhand einer Grammatik-Datei einen Syntax-Analysator in Form des \visitor Entwurfsmuster erzeugt. Diese Grammatik-Datei enthält die Spezifikation für \toya.
Anhand dieses generierten Syntax-Analysators durchläuft \toya den von ANTLR erzeugten Syntaxbaum und liefert einen abstrakten Syntaxbaum, der um Typinformationen bereichert wurde. Mithilfe des Typsystems von Kotlin unterscheidet das Backend anschließend zwischen den verschiedenen Strukturen von \toya.

Das Backend verwendet als zentrale Bibliothek ObjectWeb ASM zum Generieren von Bytecode und class-Dateien. Als erstes erzeugt \toya eine einzelne \texttt{Main}-Klasse, in welcher alle Funktionen eines \toya Programms liegen. Anschließend durchläuft das Backend den abstrakten Syntaxbaum. Jede Klasse des abstrakten Syntaxbaums besitzt eine Generierungs-Funktion, die den richtigen Bytecode für den Knoten des abstrakten Syntaxbaumes erzeugt. Sobald der abstrakte Syntaxbaum fertig durchlaufen ist, erzeugt \toya eine ausführbare class-Datei. Ist ein \toya Programm über mehrere Dateien verteilt, fasst der Compiler das Ergebnis in einer class-Datei zusammen.		
\chapter{Abstract}

\begin{english} %switch to English language rules
\Toya is a strongly typed, Turing-complete programming language for the Java Virtual Machine (JVM). \Toya allows the creation of functions, variables, arithmetic expressions, and control flow in the form of if-branches and for-loops.

Variables can be of type \texttt{int}, \texttt{double}, \texttt{string}, or \texttt{boolean}. It is also possible to create arrays of these types. With the exception of arrays, the explicit specification of the variable type is not possible. Instead, \toya determines the variable type based on the type of the assigned expression.

The compiler for \toya can be divided into two parts: frontend and backend. The frontend parses the source code to create an abstract syntax tree (AST). The backend then uses the AST to generate bytecode for the JVM. \Toya produces class files which the JVM then can process and evaluate. The compiler for \toya is completely implemented in Kotlin/JVM. 

The frontend uses the fourth version of ANTLR as its core library. ANTLR generates the parser for \toya based on a grammar file containing the language specification. This parser uses the visitor pattern to traverse the syntax tree and parse it into an AST. The AST is enriched with type information among other things. Afterwards the backend uses this type information to differentiate between the structures of \toya.

The backend uses ObjectWeb ASM as its core library to generate bytecode and class files. First of all, the compiler creates a single class called \texttt{Main} in which all the functions of the program reside. Even if a \toya program is distributed over several files the compiler still only generates a single class file. The backend then traverses the AST. Each class of the AST has a \texttt{generate} function which generates the correct bytecode for a corresponding node of the AST. As soon as the backend has finished traversing the AST the \toya compiler generates an executable class file.
\end{english}

			

%%%-----------------------------------------------------------------------------
\mainmatter                             % Hauptteil (ab hier arab. Seitenzahlen)
%%%-----------------------------------------------------------------------------

\chapter{Einleitung}
\label{cha:Einleitung}

Compiler sind ein essenzieller Grundstein der Informatik und stellen das Bindeglied zwischen Mensch und Maschine dar. Erst durch Compiler ist die Entwicklung von höheren Programmiersprachen und Programmen in einer Größenordnung möglich, die den Anforderungen des 21. Jahrhunderts gerecht werden. Ohne Compiler wäre die kommerzielle Entwicklung von Software unvorstellbar. Sie sind eine logische Konsequenz, den Anforderungen nach immer effizienter werdenden Anwendungen nachzukommen. Compiler sind die Brücke zwischen Maschinencode und dem Menschen. So weit, dass die Anweisungen an den Computer in Programmiersprachen wie SQL beinahe Englischen Sätzen entsprechen. Im Zeitalter der kollaborativen Entwicklung gilt es, Programme nicht nur zu schreiben, sondern auch für sich selbst und Kolleg:innen lesbar zu machen. All diese Aspekte laufen darauf hinaus, dass Compiler die einzige Lösung sind.

Dementsprechend ist es elementar, die Funktionsweisen und Abläufe von Compilern zu verstehen. Der beste Weg, um dies zu bewerkstelligen, ist einen Compiler selbst zu entwickeln.

\section{Zielsetzung}

Ziel dieser Arbeit ist es, eine neue Programmiersprache und einen entsprechenden Compiler dafür zu entwickeln. Dieser Compiler erzeugt Bytecode für die Java Virtual Machine, um \toya-Programme anschließend plattformunabhängig ausführen zu können. \Toya soll die Definition von Funktionen, primitiven Variablen und Feldern und Kontrollflüssen in Form von Verzweigungen und Schleifen erlauben. Daraus ergibt sich eine Turing-vollständige Programmiersprache, aus welcher heraus theoretisch alle anderen Programmiersprachen entstehen könnten. \Toya orientiert sich syntaktisch an Programmiersprachen wie C und Java mit einem Fokus auf Simplizität.

\section{Wieso die JVM?}

Eine grundsätzliche Frage, die es vor der eigentlichen Arbeit zu beantworten gilt, ist ob das Kompilat des \toya Compilers Maschinencode oder ein Zwischenprodukt in Form von Bytecode für eine virtuelle Maschine produzieren soll. Während der native Ansatz mit Maschinencode eine höhere Ausführungsgeschwindigkeit mit sich bringt, kommt der Nachteil der Implementierungskomplexität des Compilers dazu.

Virtuelle Maschinen hingegen bieten eine Abstraktionsschicht über dem Maschinencode und machen die Entwicklung eines Compilers daher wesentlich leichter. Das ermöglicht, den Fokus auf andere Aspekte, wie die Verarbeitung des Syntaxbaumes und die Generierung des Bytecodes zu legen; was auch im Verlauf dieser Arbeit klar zu erkennen ist.

Damit ist klar erkennbar, dass eine virtuelle Maschine die Ausgabe des Compilers interpretieren soll. Unklar ist jedoch weiterhin, welche virtuelle Maschine das Ziel sein soll. Aufgrund der Menge an verfügbaren Resourcen, beschränkt sich die Auswahl auf die Java Virtual Machine und Common Language Runtime. Zweiteres bringt einige Vorteile, wie zum Beispiel, dass Typinformationen generischer Typen zur Laufzeit erhalten bleiben. \Toya nutzt jedoch nur einen Bruchteil aller möglichen Eigenschaften, weswegen dies keine entscheidende Rolle spielt. Schlussendlich fällt die Entscheidung auf die Java Virtual Machine aufgrund der Familiarität damit. Eine Implementierung für die Common Language Runtime ist jedoch ebenso umsetzbar.

\section{Aufbau der Arbeit}

Die Arbeit beschäftigt sich zuerst mit den theoretischen Grundlagen von \toya und den Werkzeugen, denen \toya zugrunde liegt. Anschließend werden konkrete Implementierungsdetails präsentiert und die Funktionalität der Sprache anhand von Beispielen bewiesen.

Das Kapitel \nameref{cha:toya} umfasst die Spezifikation von \toya, ohne jedoch auf Implementierungsdetails einzugehen. Anschließend kommt es im Kapitel \nameref{cha:comparison} zur Gegenüberstellung mit einer etablierten Programmiersprache. Hierbei wurde Kotlin gewählt, da es sich dabei um die Implementierungssprache des \toya Compilers handelt. Die Kapitel \nameref{cha:antlr} und \nameref{cha:jvm} bieten eine theoretische Beschreibung von ANTLR und der JVM. Das Kapitel \nameref{cha:implementation} behandelt Implementierungsdetails des Compilers und bietet einen konkreten Einblick in die Architektur anhand von Codebeispielen und Diagrammen. \nameref{cha:tests} zeigt die Funktionalität von \toya anhand von konkreten Codebeispielen und deren Ausgaben. In \nameref{cha:Schluss} wird über das Ergebnis reflektiert und weitere Schritte erläutert.
\chapter{Die Programmiersprache toya}
\label{cha:toya}

\textit{Toya} ist eine stark typisierte, turing-vollständige Programmiersprache für die Java Virtual Machine mit einem Fokus auf Simplizität. Der Syntax ist, wie C\# oder Java zum Beispiel, stark an C angelehnt. Auf den Syntax wird in diesem Kapitel bei der näheren Behandlung der einzelnen Komponenten der Sprache eingegangen. 

\begin{CJK}{UTF8}{ipxm}
Der Name \textit{toya}, in Anlehnung an Java und Kotlin, findet seinen Ursprung bei einer Insel. Dabei handelt es sich konkret um 洞爺湖 (Tōya-ko), einen Kratersee im Norden Japans, der wiederum die Insel 中島 (Nakajima) beinhaltet. Von Tōya-ko leitet sich dann der Name \textit{toya}.
\end{CJK}

Grundsätzlich folgt toya imperativen Programmierparadigmen. Ein toya-Programm besteht aus einer Menge an Funktionen und Variablen, wobei, wie in Java eine \texttt{main} Funktion zum Programmeinstieg benötigt wird. Funktionen beinhalten verschiedene  Klassen gibt es keine; bei der Kompilation werden aber alle Programmteile in eine \texttt{Main} Klasse zusammengefasst, da jede \textit{.class} Datei genau eine Klasse beinhalten muss. Sollte die \texttt{main} Funktion nicht vorhanden sein, so wird eine Exception während des Parsens geworfen. Variablen, die außerhalb von Funktionen definiert werden, können global verwendet werden. Global in diesem Kontext bedeutet, dass Variablen in allen Funktionen des Programms verwendet werden können. 

\section{Typen}

\textit{Toya} stellt insgesamt 5 Typen und deren Array-Gegenstücke zur Verfügung. Diese sind \texttt{boolean}, \texttt{int}, \texttt{char}, \texttt{double} und \texttt{string}, wobei hierbei \texttt{int}, \texttt{boolean} und \texttt{string} eindeutig die wichtigsten sind, da jeder dieser Typen in unterschiedlichen Domänen seine Verwendung findet. So werden boolean  Das erstellen weiterer Typen ist nicht möglich.

Array-Typen werden über den allgemein bekannten Suffix `\texttt{[]}' deklariert. So ist zum Beispiel der Typ eines String-Arrays als \texttt{string[]} zu schreiben. Die JVM bietet zusätzlich noch die Datentypen \texttt{byte}, \texttt{short}, \texttt{long} und \texttt{float} an, aber weil alle dieser Typen redundant in ihrem Verwendungszweck sind, kommen diese nicht in \textit{toya} vor, da der Sinn von \textit{toya} nicht die vollständige Ausschöpfung aller JVM-Features ist, sondern die explorative Implementierung einer Programmiersprache, wofür nicht alle Datentypen benötigt werden. Der \texttt{returnAddress} Typ wird hier außer acht gelassen, weil dieser nur JVM-interne Relevanz hat.

Integer besitzt einen Wertebereich von $-2^{31}$ bis $2^{31} - 1$; Double folgt der IEEE 754 Spezifikation: 1 Bit für das Vorzeichen 11 Bit für den Exponenten und 52 Bit für die Mantisse. Boolean kann die Werte \texttt{true} und \texttt{false} annehmen. \texttt{True} wird intern als $1$ gehandhabt, \texttt{false} als 0. Die JVM besitzt keinen nativen String Typen, da dieser als Referenzwert gehandhabt wird. Da \textit{toya} keine Erstellung von Typen erlaubt, sind die einzigen Referenztypen String und Arrays. Die Bytecode-Generierung \textit{toya's} unterscheidet immer zwischen Arrays und nicht-Arrays, wenn es um die Auswahl der richtigen Opcodes geht, dadurch ergibt sich, dass eine Referenz, welche kein Array ist, immer ein String sein muss. Strings werden als Literale via doppelte Anführungszeichen definiert. So ist ein Hello World String als \texttt{``Hello World''} anzugeben.

\section{Funktionen}

Funktionen sind die zentrale Komponente von \textit{toya} und beinhalten die Programmlogik.
Sie bestehen aus einem Funktionskörper und der Funktionssignatur. Die Funktionssignatur beinhaltet Name der Funktion, Parameter und Rückgabewert. Der Name ist das einzig verpflichtende hierbei; Parameter und Rückgabewert sind rein optional. Hat eine Funktion keinen Rückgabewert, so ist in der Funktion der Pfeil, als auch der nachfolgende Typ wegzulassen.

\begin{ToyaCode}[numbers=none, caption={Eine typische Funktion unter toya.}]
function add(lhs: int, rhs: int) -> int {
    lhs + rhs
}
\end{ToyaCode}

Der Funktionskörper beinhaltet eine beliebige Menge an Statements und Ausdrücken. Hat eine Funktion einen Rückgabewert, so kann mit \texttt{return} ein nachfolgender Ausdruck rückgegeben werden. Das Schlüsselwort \texttt{return} ist jedoch optional: Wenn das letzte Statement gleichzeitig ein Ausdruck und vom gefordertem Typen ist, dann wird automatisch dieser Ausdruck retourniert. Hierbei ist jedoch aufzupassen, dass die Leserlichkeit erhalten bleibt.

Funktionen werden mit dem Syntax \texttt{<funcname>(parameter*)} aufgerufen und sind vom Rückgabetypen der aufgerufenen Funktion. Für jeden Parameter kann jeder beliebige Ausdruck eingesetzt werden, solange der Ist- und Solltyp übereinstimmt.

\section{Statements}

Statements sind Programmanweisungen, die keinen Wert zurückgeben. Dazu gehören Variablendeklaration und -zuweisung, For-Schleifen und Return-Anweisungen.

\subsection{Variablen}
\textit{Toya} erlaubt der Nutzerin die Erstellung von Variablen in Funktionen und auf globaler Ebene. Der Syntax dafür lautet \texttt{var <name> = <ausdruck>}; die explizite Angabe eines Typens ist nicht möglich. Stattdessen inferiert der Compiler anhand bekannter Typinformation des zu evaluierenden Ausdrucks den Typen und weist diesen Typen der Variable zu. Dieser Typ bleibt über die gesamte Lebensdauer der Variable gleich. Sobald der Typ einer Variable einmal fixiert ist, so kann dieser Typ nicht mehr geändert werden. Initialisiert man also eine Variable mithilfe eines Ausdrucks, der zu \texttt{int} evaluiert, so ist die Variable bis zur Vernichtung durch den Garbage Collector vom Typen \texttt{int}. Eine getrennte Deklaration und Initialisierung ist nicht möglich.

Abgesehen von typentheoretischer Relevanz bietet die Verwendung des Schlüsselwortes \texttt{var} einige Vorteile als auch Nachteile für Verwenderinnen von \textit{toya}. Da \texttt{var} eine Vielzahl von verschiedenen Typen ersetzt, erleichtert es die Schreibarbeit für Programmiererinnen ungemein. Robert C. Martin sagt jedoch in seinem nominalen Werk \textit{A handbook of agile software craftmanship} ``Code is more read than it is written.''. Daraus folgt, dass die Lesbarkeit wichtiger als \textit{Schreibbarkeit} von Code ist und hierbei zeigen sich dann auch die Schwächen.

Ohne einer modernen Entwicklungsumgebung, welche via dem User-Interfaces Hinweise auf die Typisierung gibt, kann die Entwicklerin nur über den konkreten Typen Vermutungen anstellen. Aufgrund der geringen Anzahl an Typen in \textit{toya} ist das Fehlen von Typhinweisen jedoch vernachlässigbar. Vergleicht man nun die Variablendeklaration und Initialisierung mit Java, so ist zu erkennen, dass bei der Initialisierung via Literalen die Verwendung von \texttt{var} kein Problem darstellt. Will man einer Variable den retournierten Wert einer Funktion zuweisen, so können hier aber Schwierigkeiten hinsichtlich Schlussfolgerungen auftreten. Daher ist die bewusste und intelligente Vergabe von Variablennamen essenziell.

\begin{ToyaCode}[numbers=none, caption={Variablendeklaration in toya}]
var number = 123
var word = "Hello World"
var value = 123.456
var bool = true
var result = someFunction()
\end{ToyaCode}

\begin{JavaCode}[numbers=none,caption={Variablendeklaration in Java (vor Version 10)}]
int number = 123;
String word = "Hello World";
double value = 123.456;
boolean bool = true;
int result = someFunction();
\end{JavaCode}

Der Name einer Variable darf nur einmal im eigenen Scope oder Elternscope verwendet werden, da es ansonsten zu Unklarheiten kommen kann, welche von mehreren Variablen nun gemeint ist. Auf die Frage, was sich hinter einem Scope verbiergt, wird näher im Kapitel \ref{cha:implementation} eingegangen. Ein Name darf aus beliebig vielen Groß- und Kleinbuchstaben und Unterstrichen bestehen. 

\subsection{Arrays}

Arrays sind eine eindimensionale Liste von einem bestimmten Typen mit einer fixen Länge, welche bei der Deklaration des Arrays angegeben wird und sich über die Lebensdauer der Variable nicht mehr ändert. Zuweisungen und Deklarationen unterscheiden sich nur leicht von nicht-Array Variablen. Der Hauptunterschied dabei besteht darin, dass die Länge bei der Deklaration und der Index bei der Zuweisung angegeben werden muss.

Arrays werden mit dem Syntax \texttt{var <name> = new <typ>[int-Ausdruck]} deklariert. Mit \texttt{<name>[int-Ausdruck] = Ausdruck} wird ein neuer Wert auf einen Index geschrieben.

\subsection{For-Schleifen}

\section{Kommentare}

\section{Ausdrücke}

\subsection{Arithmetik}

\subsection{Boolsche Logik}

\subsection{If-Ausdrücke}

\textit{Toya} erlaubt die Erstellung von Kommentaren. Ein Kommentar beginnt mit einem doppelten Schrägstrich \texttt{//}; alle weiteren Zeichen in dieser Zeile -- bis zum Zeilenumbruch -- ignoriert der Parser.

- introduction von toya (grundsätzliche features, "auf diese wird in diesem Kapitel noch näher eingegangen")
- namenserklärung
- Funktionen
- Typen
- Variablen
- Expressions:
    - Arithmetik
    - Boolean
    - If
    - Arrays
- Statements
    - For Statement
\chapter{Vergleich mit Kotlin}
\label{cha:comparison}

Da der \toya-Compiler in Kotlin impelementiert und die Syntax in einigen Aspekten auch an Kotlin angelehnt ist liegt es nahe, einen Vergleich zwischen den beiden Sprachen durchzuführen. Ein besonderer Fokus im Vergleich liegt auf der Syntax, da der generierte Bytecode zwischen den beiden Sprachen kaum zu unterscheiden ist (abgesehen von Optimierungen für individuelle Code-Stücke in Kotlin). Im Folgenden werden nur die wesentlichen Unterschiede zwischen den beiden Sprachen im Umfang von \toya aufgezeigt. Im Sinne der Prägnanz wird über den direkten Vergleich hinaus nicht näher auf Kotlin eingegangen.

\section{Übersicht Kotlin}
Kotlin ist eine seit 2011 entwickelte statisch typisierte, imperative Programmiersprache mit Elementen der funktionalen Programmimerung. Als Antwort von JetBrains auf Java und Scala bietet Kotlin eine plattformübergreifende und idiomatische Programmiersprache, welche nahtlos in das JVM Ökosystem eingegliedert ist. Neben der JVM kompiliert Kotlin auch auf JavaScript, WebAssembly und Assembly. Für letzteres verwendet Kotlin die Compiler-Infrastruktur LLVM. Das momentane Haupt-Anwendungsgebiet liegt in der mobilen Entwicklung unter Android. Google empfiehlt seit 2019 [TODO: Google I/O 2019 reference https://developer.android.com/kotlin/first] Kotlin anstatt Java zu verwenden und bietet Teile der Standardbibliothek - Jetpack Compose - nur noch unter Kotlin an, da hierbei der Einsatz von Kotlin-spezifischen Compiler Plugins notwendig ist. Andrey Breslav leitete bis 2020 die Entwicklung von Kotlin und übergab anschließend die Verantwortung an Roman Elizarov. Aktuell befindet sich Kotlin bei der Version 1.8.10.

Eines der wichtigsten Merkmale von Kotlin ist die Vermeidung des sogenannten "Billion Dollar Mistake": Null Pointer. Indem der Compiler bei der unerlaubten Zuweisung von Null-Werten in einen Ausnahmezustand versetzt wird, kann es während der Laufzeit nicht mehr zu unerwünschtem Verhalten kommen. Da es jedoch weiterhin Fälle gibt, in welchen null ein erwünschter Wert ist, kann die Entwickler:in via \texttt{?} den Typ einer Variable als \textit{nullable} markieren.

Eines weiteres wichtiges Ziel Kotlins ist die Lesbarkeit des Codes. Dies zeigt sich vor allem in den Methoden der Standardbibliothek. So gibt es zum Überprüfen von Listen auf deren Leerheit die Methode \texttt{isEmpty()}, aber auch dessen Negation mit \texttt{isNotEmpty()}. Diese zweite Methode ist redundant, aber erleichtert die Lesbarkeit des Codes ungemein.

\section{Funktionen}

\section{Variablen}

In Kotlin stehen zur Deklaration von Variablen die Schlüsselwörter \texttt{var} und \texttt{val} zur Verfügung. Die Bestimmung des Typs erfolgt entweder implizit anhand des zugewiesenen Ausdrucks oder explizit. \texttt{val someString: String = "Hello World"} weißt zum Beispiel der unveränderlichen Variable \textit{someString} den Wert \textit{Hello World} zu. Die explizite Angabe des Typs \textit{String} ist bei diesem Beispiel redundant, da aus dem Ausdruck der Typ ableitbar ist und daher nicht zwingend notwendig.

\toya Im Gegensatz dazu bietet \toya zur Variablendeklaration nur das Schlüsselwort \texttt{var} an, da in \toya alle Variablen veränderbar sind. Die explizite Angabe von Typen ist nicht möglich. Der Typ leitet sich immer vom Ausdruck ab.

\section{Felder}

Kotlin bietet für die Verwendung von Felder die generische Klasse \texttt{Array} an. Da Felder, wie in anderen Sprachen, auch in Kotlin statisch in ihrer Größe sind, ist die Anzahl an speicherbaren Elementen als Konstruktorparameter anzugeben. Alternativ dazu besteht die Möglichkeit, mithilfe der Hilfs-Funktion \texttt{arrayOf(...)} ein Feld mit Werten zu initialisieren. \toya hingegen beruft sich auf den C-artigen Syntax und verwendet die eckigen Klammern \texttt{[]} zur Initialisierung von Feldern. Die Größe ist abermals statisch und als Ausdruck innerhalb der eckigen Klammern anzugeben.

\section{If-Verzweigung}

Die Semantik und Syntax von If-Verzweigungungen in \toya gleichen denen in Kotlin. So kann die Benutzer:in sowohl einen einzelnen Ausdruck als auch einen gesamten Programmblock als Zweig angeben. Ebenso ersetzt die normale If-Verzweigung den ternären Operator wenn alle Zweige einen Ausdruck darstellen. Dies ist ein weiteres Mittel von Kotlin, die Lesbarkeit des Codes zu verbessern. 

\section{For-Schleifen}

Während \toya For-Schleifen denen von Java stark ähneln, bietet Kotlin eine deutlich kompaktere, aber auch flexiblere Syntax an. So hat die Benutzer:in die Möglichkeit entweder über einen \texttt{int}-Bereich oder auch über eine Enumeration zu iterieren. 

\begin{KotlinCode}[numbers=none, caption={Einfache For-Schleife in Kotlin}]
for (i in 0..10 step 2) {
    print(i)
}
\end{KotlinCode}
\chapter{Generierung des Syntaxbaums}
\label{cha:antlr}

ANTLR (\textbf{AN}other \textbf{T}ool for \textbf{L}anguage \textbf{R}ecognition) ist ein seit 1989 entwickeltes Werkzeug von Terrence Parr zum Erzeugen von Parsern, Lexern und Compilern. Die Benutzer:in definiert diese mit einer Grammatik und erzeugt daraus dann mithilfe eines von ANTLR zur Verfügung gestelltem Kommandozeilen-Werkzeug Parser und Lexer in der gewünschten Ziel-Sprache. ANTLR unterstützt unter anderem Java, C\#, Python, JavaScript, Go, C++, Swift, PHP und Dart. \toya bietet als aktuellste Version 4 an, welche jedoch große Unterschiede - allen voran der Umstieg auf eine effizientere Parsing-Methodik - zu Version 3 bietet.

ANTLR findet auch im professionellen Umfeld Verwendung. So verwendet Twitter ANTLR zur Syntax-Analyse von über 2 Milliarden Suchanfragen pro Tag. Hadoop verwendet ANTLR zur Syntax-Analyse von \textit{Hive} und \textit{Pig} und Netbeans analysiert den Syntax von \textit{C++} mithilfe ANTLR. 

Adaptive LL(*) Parsing stellt den größten Unterschied zwischen Version drei und vier von ANTLR dar. ALL(*) ist ein neuer von Terrence Parr entwickelter Parsing-Ansatz, welcher theoretisch zwar eine Laufzeitkomplexität von $\mathcal{O}(n^4)$ hat, praktisch gesehen aber lineare Performanz aufweist. Im Gegensatz zum traditionellen LL- oder LR-Parsing, das auf einer vordefinierten Grammatik basiert, analysiert ALL(*) die gesamte Eingabe, um das Parsing-Entscheidungsdiagramm zu konstruieren, das zur Analyse der PEG verwendet wird. ALL(*) verwendet eine Technik namens adaptive Vorwärtsanalyse, um den automatisch erzeugten Parsing-Entscheidungsbaum zu verbessern. Diese Technik kombiniert Vorwärts- und Rückwärtsanalyse, um die Vorhersage der nächsten Token zu verbessern.

Der große Vorteil von ANTLR gegenüber selbst entwickelten Lösungen zum Erzugen von Syntaxbäumen ist die Effizienz mit welcher neue Grammatikregeln definiert werden können. Diese Effizienz und Leichtigkeit in der Umsetzung hat jedoch auch seine Kosten. Da ANTLR einen umfangreichen Syntaxbaum erzeugt und es sehr unwahrscheinlich ist, dass das Programm alle Daten des Syntaxbaums benötigt, kommt es zu einem Mehraufwand für Daten, die keinen Nutzen finden.

Deswegen verwenden alle größeren Programmiersprachen (C++, Python, C\#, Java, etc.) selbst entwickelte Lexer und Parser um eine substantielle Reduktion der Übersetzungszeiten zu erreichen.

Da \toya über eine experimentelle Programmiersprache nicht hinaus geht und es den programmatischen Aufwand sprengt, wurde aktiv gegen eine maßgeschneiderte Lösung für \toya entschieden. Die Übersetzungszeiten unter Verwendung von ANTLR sind im Rahmen von \toya akzeptabel. Lexer und Parser, die durch ANTLR erzeugt werden sind nicht schnell, aber schnell genug.

Nun stellt sich die Frage: Wieso nicht auf reguläre Ausdrücke zurückgreifen? Dafür existieren mehrere Gründe:
\begin{itemize}
    \item Kein rekursives Parsing möglich.
    \item Programmelemente, die an allen Stellen im Code - Kommentare zum Beispiel - auftauchen können, sind an allen potenziellen Stellen im Regex-Ausdruck zu berücksichtigen. Dies führt zu Redundanz.
    \item Reguläre Ausdrücke wachsen schnell und sind schwer zu verwalten. Da Programmmiersprache typischerweise auch in ihrem Funktionsumfang wachsen, sind reguläre Ausdrücke nicht dafür geeignet und führen zu einer schlechten Skalierbarkeit.
\end{itemize}

% Welche Sprachen unterstützt https://github.com/antlr/antlr4/blob/master/doc/targets.md
% Differences Antlr 3 und 4: https://github.com/antlr/antlr4/blob/master/doc/faq/general.md

\section{Grammatik-Definition}

Die Grammatik-Definition in einer \texttt{g4}-Datei ist der Ausgangspunkt für alle weiteren Schritte in ANTLR. Diese Datei beinhaltet alle Regeln für den Lexer und Parser und Meta Informationen anhand welcher Eingabe-Datein abzuarbeiten sind. Meta Informationen befinden sich typischerweise am Beginn der Datei.

Da Leerzeichen in der Regel unwichtig sind und und keine Relevanz für die Semantik der Sprachen haben (abgesehen von Ausnahmefällen wie Python), ignoriert man diese. Dies erfolgt mithilfe der Anweisung \texttt{\string[ \textbackslash t\textbackslash n\textbackslash r\string]+ -> skip}, welche angibt, dass Leerzeichen bei der Abarbeitung einer Eingabedatei zu überspringen sind.
%  Diese Syntax findet auch Einsatz bei \textit{Channels}. Anstatt, wie im Falle von Leerzeichen, die Zeichen wegzuwerfen, 

Ob eine Regel den Parser oder Lexer betrifft, hängt vom Anfangsbuchstaben dieser Regel ab. Ist das erste Zeichen ein Großbuchstabe, betrifft es den Lexer; wenn nicht, den Parser. Typischerweise werden als Erstes Regeln für den Parser und als Zweites Regeln für den Lexer definiert. Die Reihenfolge der Lexer-Regeln ist von Relevanz, da in derselben Reihenfolge ANTLR diese Regeln analyisiert.

\begin{AntlrCode}[numbers=none, caption={Beispielhafter Aufbau einer Grammatik-Definition für Additionen}]
grammar: addition;

// Parser Regeln
operation  : NUMBER '+' NUMBER ;

// Lexer Regeln
NUMBER     : [0-9]+ ;
WHITESPACE : [ \t\n\r]+ -> skip ;
\end{AntlrCode}

\section{Listener}

Um die Ergebnisse des Syntaxbaums abarbeiten zu können, bietet ANTLR zwei Entwurfsmuster an: \visitor und \listener. Die Implementierung dieser Entwurfsmuster erzeugt die Benutzer:in mithilfe des Kommandozeilen-Werkzeug \texttt{antlr4} anhand der anzugebenden Grammatik-Datei. Außerdem gibt die Benutzer:in zusätzlich noch an, ob entweder die Implementierung für Visitor oder Listener oder für Beide zu generieren sind. Sollen keine \listener generiert werden, ist dies via dem Argument \texttt{-no-listener} anzugeben.

\listener haben im Gegensatz zum \visitor keinen Einfluss auf den Analyse-Vorgang. Stattdessen ruft der \textit{Tree Walker}, der den Syntaxbaum traversiert, die von ANTLR generierten Methoden für den richtigen Knotentyp anhand der Analyse-Regeln auf. Diese Methoden des \listeners liefern keinen Wert zurück, was die Verwaltung eines abstrakten Syntaxbaums erschwert. Aufgrund der Komplexität von \toya sind \listener daher nicht empfehlenswert.

\listener bieten sich gut für Zusatzverhalten an, welches den Syntaxbaum nicht verändert. Typische Verwendungszwecke für \listener ist das Loggen von Informationen oder ermitteln von Metadaten.

\section{Visitor}

Das \visitor Pattern ist ein Entwurfsmuster, das es ermöglicht, eine Operation auf den Elementen einer Objektstruktur auszuführen, ohne die Klassen dieser Elemente zu ändern oder die Struktur selbst zu verändern.

Das Entwurfsmuster besteht aus zwei grundlegenden Komponenten: den Elementen der Objektstruktur und dem Visitor, der die Operation auf den Elementen ausführt. Die Elemente der Struktur implementieren eine gemeinsame Schnittstelle, das den \visitor akzeptiert. Der \visitor selbst definiert Methoden für jede Klasse von Elementen, die er besuchen kann.

Um das Entwurfsmuster zu nutzen, ruft man die \texttt{accept}-Methode des \visitors auf dem Wurzelelement der Struktur auf, welches die Schnittstelle für die Elemente implementiert. Diese Methode wiederum ruft die entsprechende Methode im \visitor auf, wodurch dieser das Element besucht. Das Element gibt sich selbst als Parameter an den \visitor weiter, wodurch dieser auf die Eigenschaften und Methoden des Elements zugreifen und eine Operation darauf ausführen kann. Ein mögliches Problem hierbei ist, dass Fehler leichter enstehen können. Vergisst die Entwickler:in auf den Aufruf einer notwendigen \texttt{accept}-Methode, kommt es nicht zur Versetzung des Programms in einen Ausnahmezustand, sondern die zu parsenden Token werden ignoriert.

Das \visitor Entwurfsmuster hat den Vorteil, dass es das Open-Closed-Prinzip unterstützt, da neue Operationen durch die Erstellung neuer \visitor-Klassen hinzugefügt werden können, ohne die existierenden Elementklassen zu ändern. Außerdem können komplexe Operationen auf der Objektstruktur durchgeführt werden, indem man mehrere \visitor-Klassen erstellt, die jeweils eine Teiloperation durchführen.

\begin{ToyaCode}
class ExpressionVisitor(val scope: Scope) : toyaBaseVisitor<Expression>() {
    override fun visitValue(valueContext: toyaParser.ValueContext): Expression {
        val value = valueContext.text
        val type = TypeResolver.getFromValue(value)
        return Value(value, type)
    }
    // Rest of class...
}
\end{ToyaCode}
\chapter{JVM und Bytecode}
\label{cha:jvm}

die Java Virtual Machine (JVM) ist eine von Sun konzipierte und in Folge von Oracle weiterentwickelte Stack-basierte virtuelle Maschine, die die Ausführung von Bytecode unter Linux, MacOS und Windows ermöglicht (TODO Full Lineup checken.)
\chapter{Generierung des Bytecodes mit ASM}
\label{cha:asm}
\chapter{Implementierung von toya}
\label{cha:implementation}

Lorem ipsum dolor sit amet.
\chapter{Tests}
\label{cha:tests}

Um die Funktionalität von \toya zu gewährleisten, sind Tests zu vollziehen. Diese Tests verlaufen in drei Schritte, wie folgt:
\begin{enumerate}
    \item Programmcode in \toya schreiben
    \item den kompilierten Bytecode analysieren
    \item Die Ausgabe des Programms überprüfen
\end{enumerate}

Die Analyse des Bytecode erfolgt mit dem, in der JDK inkludierten Werkzeug \texttt{javap}. Dieses erlaubt es, den Bytecode einer \texttt{class}-Datei in einer für Menschen leserlicher Form auszugeben. Der Aufruf erfolgt über die Kommandozeile im Format: \texttt{javap -c Main.class}. Das Argument \texttt{-c} zeigt zusätzlich die Bytecode Befehle innerhalb der Methoden an. Die Ausführung der Programme erfolgt über die Kommandozeile mit dem Befehl \texttt{java Main}.

Insgesamt gilt es, die einzelnen Sprachkonstrukte, wie zum Beispiel For-Schleifen und Variablendeklaration und anschließend umfangreichere Programme zu testen.

\section{Hello World}

\begin{ToyaCode}[numbers=none, caption={Quelltext des Hello World Programms}]
function main(args: string[]) {
    print("Hello World")
}
\end{ToyaCode}

\begin{JavaCode}[numbers=none, caption={Bytecode des Hello World Programms}]
public class Main {
    public static void main(java.lang.String[]);
        Code:
            0: getstatic     #12    // Field java/lang/System.out:Ljava/io/PrintStream;
            3: ldc           #14    // String Hello World
            5: invokevirtual #20    // Method java/io/PrintStream.println:(Ljava/lang/String;)V
            8: return
}
\end{JavaCode}

\begin{ToyaCode}[numbers=none, caption={Konsolen-Ausgabe des Hello World Programms}]
Hello World    
\end{ToyaCode}

\section{Funktionen}

\begin{ToyaCode}[numbers=none, caption={Funktionen}]
function title() {
    print("This is an addition:")
}

function add(lhs: int, rhs: int) -> int {
    lhs + rhs
}

function main(args: string[]) {
    title()
    print(add(1,2))
}
\end{ToyaCode}  

\begin{JavaCode}[numbers=none, caption={Bytecode für Funktionen}]
public class Main {
    public static void title();
        Code:
            0: getstatic     #12    // Field java/lang/System.out:Ljava/io/PrintStream;
            3: ldc           #14    // String This is an addition:
            5: invokevirtual #20    // Method java/io/PrintStream.println:(Ljava/lang/String;)V
            8: return

    public static int add(int, int);
        Code:
            0: iload_0
            1: iload_1
            2: iadd
            3: ireturn

    public static void main(java.lang.String[]);
        Code:
            0: invokestatic  #26    // Method title:()V
            3: getstatic     #12    // Field java/lang/System.out:Ljava/io/PrintStream;
            6: ldc           #27    // int 1
            8: ldc           #28    // int 2
            10: invokestatic  #30   // Method add:(II)I
            13: invokevirtual #33   // Method java/io/PrintStream.println:(I)V
            16: return
}    
\end{JavaCode}

\begin{ToyaCode}[numbers=none, caption={Konsolen-Ausgabe der Funktionen}]
This is an addition:
3  
\end{ToyaCode}


\section{Variablen}

\begin{ToyaCode}[numbers=none, caption={Variablen}]
function someFunction() -> int {
    return 8
}

function main(args: string[]) {
    var number = 123
    var word = "Hello World"
    var double = 123.456
    var bool = true
    var result = someFunction()

    print(number)
    print(word)
    print(double)
    print(bool)
    print(result)
}
\end{ToyaCode}

\begin{JavaCode}[numbers=none, caption={Bytecode von Variablen}]
public class Main {
    public static int someFunction();
        Code:
            0: ldc           #7                  // int 8
            2: ireturn
    
    public static void main(java.lang.String[]);
        Code:
            0: ldc           #10    // int 123
            2: istore_1
            3: ldc           #12    // String Hello World
            5: astore_2
            6: ldc2_w        #13    // double 123.456d
            9: dstore_3
            10: iconst_1
            11: istore        5
            13: invokestatic  #16   // Method someFunction:()I
            16: istore        6
            18: getstatic     #22   // Field java/lang/System.out:Ljava/io/PrintStream;
            21: iload_1
            22: invokevirtual #28   // Method java/io/PrintStream.println:(I)V
            25: getstatic     #22   // Field java/lang/System.out:Ljava/io/PrintStream;
            28: aload_2
            29: invokevirtual #31   // Method java/io/PrintStream.println:(Ljava/lang/String;)V
            32: getstatic     #22   // Field java/lang/System.out:Ljava/io/PrintStream;
            35: dload_3
            36: invokevirtual #34   // Method java/io/PrintStream.println:(D)V
            39: getstatic     #22   // Field java/lang/System.out:Ljava/io/PrintStream;
            42: iload         5
            44: invokevirtual #37   // Method java/io/PrintStream.println:(Z)V
            47: getstatic     #22   // Field java/lang/System.out:Ljava/io/PrintStream;
            50: iload         6
            52: invokevirtual #28   // Method java/io/PrintStream.println:(I)V
            55: return
}      
\end{JavaCode}

\begin{ToyaCode}[numbers=none, caption={Konsolen-Ausgabe der Variablen}]
123
Hello World
123.456
true
8    
\end{ToyaCode}

\section{Felder}

\begin{ToyaCode}[numbers=none, caption={Felder}]
function main(args: string[]) {
    var arr = new int[8]
    arr[3] = 15

    print(arr[3])
    print(arr[4])
}
\end{ToyaCode}

\begin{JavaCode}[numbers=none, caption={Bytecode von Felder}]
public class Main {
    public static void main(java.lang.String[]);
        Code:
             0: ldc           #7    // int 3
             2: ldc           #8    // int 4
             4: if_icmpgt     11
             7: iconst_0
             8: goto          12
            11: iconst_1
            12: ifne          20
            15: ldc           #9    // int 6
            17: goto          22
            20: ldc           #10   // int 5
            22: istore_1
            23: getstatic     #16   // Field java/lang/System.out:Ljava/io/PrintStream;
            26: iload_1
            27: invokevirtual #22   // Method java/io/PrintStream.println:(I)V
            30: ldc           #7    // int 3
            32: ldc           #8    // int 4
            34: if_icmplt     41
            37: iconst_0
            38: goto          42
            41: iconst_1
            42: ifne          56
            45: getstatic     #16   // Field java/lang/System.out:Ljava/io/PrintStream;
            48: ldc           #24   // String false branch
            50: invokevirtual #27   // Method java/io/PrintStream.println:(Ljava/lang/String;)V
            53: goto          64
            56: getstatic     #16   // Field java/lang/System.out:Ljava/io/PrintStream;
            59: ldc           #29   // String true branch
            61: invokevirtual #27   // Method java/io/PrintStream.println:(Ljava/lang/String;)V
            64: return
}       
\end{JavaCode}

\begin{ToyaCode}[numbers=none, caption={Konsolen-Ausgabe der Felder}]
15
0    
\end{ToyaCode}

\section{If-Verzweigungen}

\begin{ToyaCode}[numbers=none, caption={If-Verzweigungen}]
function main(args: string[]) {
    var value = if (3 > 4) 5 else 6
    print(value)

    if(3 < 4) {
        print("true branch")
    } else {
        print("false branch")
    }
}
\end{ToyaCode}

\begin{JavaCode}[numbers=none, caption={Byteocde der If-Verzweigungen}]
public class Main {
    public static void main(java.lang.String[]);
        Code:
             0: ldc           #7    // int 3
             2: ldc           #8    // int 4
             4: if_icmpgt     11
             7: iconst_0
             8: goto          12
            11: iconst_1
            12: ifne          20
            15: ldc           #9    // int 6
            17: goto          22
            20: ldc           #10   // int 5
            22: istore_1
            23: getstatic     #16   // Field java/lang/System.out:Ljava/io/PrintStream;
            26: iload_1
            27: invokevirtual #22   // Method java/io/PrintStream.println:(I)V
            30: ldc           #7    // int 3
            32: ldc           #8    // int 4
            34: if_icmplt     41
            37: iconst_0
            38: goto          42
            41: iconst_1
            42: ifne          56
            45: getstatic     #16   // Field java/lang/System.out:Ljava/io/PrintStream;
            48: ldc           #24   // String false branch
            50: invokevirtual #27   // Method java/io/PrintStream.println:(Ljava/lang/String;)V
            53: goto          64
            56: getstatic     #16   // Field java/lang/System.out:Ljava/io/PrintStream;
            59: ldc           #29   // String true branch
            61: invokevirtual #27   // Method java/io/PrintStream.println:(Ljava/lang/String;)V
            64: return
}
\end{JavaCode}

\begin{ToyaCode}[numbers=none, caption={Konsolen-Ausgabe der If-Verzweigungen}]
6
true branch    
\end{ToyaCode}

\section{For-Schleifen}

\begin{ToyaCode}[numbers=none, caption={For-Schleifen}]
function main(args: string[]) {
    var n = 200

    for (var i = 0; i <= n; i = i+10) {
        print(i)
    }
}
\end{ToyaCode}

\begin{JavaCode}[numbers=none, caption={Bytecode der For-Schleife}]
public class Main {
    public static void main(java.lang.String[]);
        Code:
             0: ldc           #7    // int 200
             2: istore_1
             3: ldc           #8    // int 0
             5: istore_2
             6: iload_2
             7: iload_1
             8: if_icmple     15
            11: iconst_0
            12: goto          16
            15: iconst_1
            16: ifeq          34
            19: getstatic     #14   // Field java/lang/System.out:Ljava/io/PrintStream;
            22: iload_2
            23: invokevirtual #20   // Method java/io/PrintStream.println:(I)V
            26: iload_2
            27: ldc           #21   // int 10
            29: iadd
            30: istore_2
            31: goto          6
            34: return
}      
\end{JavaCode}

\begin{ToyaCode}[numbers=none, caption={Konsolen-Ausgabe der For-Schleife}]
0
10
i is 20:
20
30
40
50
60
70
80
90
100
110
120
130
140
150
160
170
180
190
200
\end{ToyaCode}

\section{Umfangreicheres Beispiel}
Die Tests bisher beschränkten sich auf einzelne Eigenschaften von \toya. Um aber auch das Zwischenspiel von mehreren Eigenschaften zu testen, wird nun noch ein umfangreicheres Programm analysiert. Dieses Programm ruft eine Funktion auf, welche Zeichenketten auf der Konsole ausgibt, definiert Variablen von Felder und primitiven Datentypen, iteriert über eine For-Schleife, welche eine If-Verzweigung enthält und verändert und liest Felder.

\begin{ToyaCode}[numbers=none, caption={Quelltext des umfangereicheren Beispiels}]
function printIntro(n: int) {
    print("Welcome to a simple toya program")
    print("Calculating numbers up to: ")
    print(n)
    print("---------")
}

function main(args: string[]) {
    var n = 200
    printIntro(n)

    for (var i = 0; i <= n; i = i+10) {
        if (i == 20) {
            print("i is 20:")
        }
        print(i)
    }

    var boolArr = new boolean[2]
    boolArr[1] = true
    print(boolArr[0])
    print(boolArr[1])
}
\end{ToyaCode}

Da das Anzeigen des Bytecodes bei diesem umfangereicheren Beispiel zu lange ist und alle Fragmente in den anderen Tests bereits zu sehen sind, wird der Bytecode an dieser Stelle ausgelassen.

\begin{ToyaCode}[numbers=none, caption={Konsolen-Ausgabe des umfangereicheren Beispiels}]
Welcome to a simple toya program
Calculating numbers up to:
200
---------
0
10
i is 20:
20
30
40
50
60
70
80
90
100
110
120
130
140
150
160
170
180
190
200
false
true    
\end{ToyaCode}
\chapter{Zusammenfassung, Schlüsse und Lehren}
\label{cha:Schluss}

% Über Testbarkeit reden -> programmfragmente als Unit-Tests implementieren. Also z.b. source code von array def als unit-test etc. -> ermöglicht leichtes Testen.

% Im großen ganzen ist \toya in vielen Aspekten eine suboptimale Lösung, die in einer \textit{Version 2.0} vieles anders umgesetzt haben würde, aber sie funktioniert.

% ANTLR sehr nice, auch wenn die nachteile klar sind. grammar definition super. visitor pattern sehr gut verwendbar. generell ANTLR sehr gutes API design. would reuse this library
% in zukunft byte buddy statt asm? mehr abstraktion von vorteil
% sehr lehrreich tho
% bei weiterentwicklung parser selbst schreiben.
% Einblick in Bytecode gibt mehr Verständniss hinsichtlich Abstraktion von Java

Ziel dieser Bachelorarbeit war es, sich mit der Implementierung einer eigens konzipierten Programmiersprache auseinanderzusetzen. Dieser Aufgabenstellung konnte erfolgreich durch die Implementierung von \toya und dessen Compiler nachgegangen. \Toya bietet in finaler Form die Definition von Funktionen, Variablen, Kontrollflüssen und arithmetischen Ausdrücken. \Toya ist statisch typisiert und bietet als Datentypen Ganz- und Gleitkommazahlen, Zeichenketten und Boole'sche Werte zur Initialisierung von Variablen, Funktionsparametern und Rückgabewerten.

Positiv hervorzuheben ist die Arbeit mit ANTLR. Nicht aufgrund der Effizienz, sondern aufgrund der ausgezeichneten Eingliederung in den Arbeitsablauf stellt es sich als besonders gutes Entwicklungswerkzeug dar. Mithilfe eines Intellij IDEA Plugins können Textfragmente hinsichtlich Ihrer Validität für eine spezifizierte Grammatik verifiziert und mögliche Fehler umgehend erkannt und behoben werden. Die Grammatikdefinition anhand g4-Dateien ermöglicht eine klare Übersicht darüber, wie die Programmiersprache nun konkret strukturiert ist.

\Toya und dessen Komponenten können auch leicht getestet werden. Als Eingabewert dienen Sprachfragmente, wie zum Beispiel eine Variablendeklaration. Als Ausgabewert, welcher anschließend auf Validität des Ergebnisses zu vergleichen ist, kann der generierte Bytecode verwendet werden. Durch diese textuelle Ein- und Ausgabe wird \toya zur leicht testbaren Sprache. Wächst nun der Umfang von \toya, so steigt der Testaufwand nicht überproportional, sondern beschränkt sich auf die neuen Sprachaspekte.

Sollte Interesse an der Weiterentwicklung von \toya bestehen, wäre es sinnvoll, ANTLR durch einen eigens implementierten Analysator zu ersetzen, da der Zeitaufwand des vollständigen Syntaxbaums mit steigender Komplexität erheblich zunimmt. Ein eigens implementierter Analysator hingegen kann viele Teile des erzeugten Syntaxbaums völlig verwerfen, da für gewöhnlich nur ein Teil aller Informationen für die schlussendliche Erzeugung des Maschinencodes relevant ist. Für weniger umfangreiche Grammatiken macht ANTLR jedoch völlig Sinn und ist daher ohne Bedenken weiterzuempfehlen.

%%%-----------------------------------------------------------------------------
% \appendix                                                               % Anhang 
%%%-----------------------------------------------------------------------------

% \include{back/anhang_a}	% Technische Ergänzungen
% \include{back/anhang_b}	% Inhalt der CD-ROM/DVD
% \include{back/anhang_c}	% Chronologische Liste der Änderungen
% \include{back/anhang_d}	% Quelltext dieses Dokuments

%%%-----------------------------------------------------------------------------
\backmatter                          % Schlussteil (Quellenverzeichnis und dgl.)
%%%-----------------------------------------------------------------------------

\MakeBibliography[nosplit] % Quellenverzeichnis

%%%-----------------------------------------------------------------------------
% Messbox zur Druckkontrolle
%%%-----------------------------------------------------------------------------

\include{back/messbox}

%%%-----------------------------------------------------------------------------
\end{document}
%%%-----------------------------------------------------------------------------
