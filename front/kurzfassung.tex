\chapter{Kurzfassung}

\Toya ist eine stark typisierte, Turing-vollständige Programmiersprache für die \textit{Java Virtual Machine} (JVM). \Toya erlaubt die Erstellung von Funktionen, Variablen, arithmetischen Ausdrücken und Kontrollstrukturen in Form von If-Verzweigungen und For-Schleifen.

Variablen können vom Typ \texttt{int}, \texttt{double}, \texttt{string} oder \texttt{boolean} sein. Ebenso ist es möglich, Felder dieser Typen anzulegen. Die explizite Angabe des Variablentyps ist mit Ausnahme von Feldern nicht möglich. Stattdessen ermittelt \toya anhand des zugewiesenen Ausdrucks den Typ der Variable.

Der \toya Compiler ist in zwei Teile unterteilt: Frontend und Backend. Das Frontend kümmert sich um die Verarbeitung des Quelltextes um daraus einen abstrakten Syntaxbaum zu erzeugen. Das Backend erzeugt anschließend anhand des abstrakten Syntaxbaums Bytecode für die JVM. Als Produkt liefert \toya class-Dateien, die die JVM verarbeitet und auswertet. Der Compiler für \toya ist vollständig in Kotlin/JVM implementiert.

Das Frontend verwendet als zentrale Bibliothek ANTLR in der vierten Version. ANTLR ist ein Syntax-Analysator-Generator, der anhand einer Grammatik-Datei einen Syntax-Analysator in Form des \visitor Entwurfsmuster erzeugt. Diese Grammatik-Datei enthält die Spezifikation für \toya.
Anhand dieses generierten Syntax-Analysators durchläuft \toya den von ANTLR erzeugten Syntaxbaum und liefert einen abstrakten Syntaxbaum, der um Typinformationen bereichert wurde. Mithilfe des Typsystems von Kotlin unterscheidet das Backend anschließend zwischen den verschiedenen Strukturen von \toya.

Das Backend verwendet als zentrale Bibliothek ObjectWeb ASM zum Generieren von Bytecode und class-Dateien. Als erstes erzeugt \toya eine einzelne \texttt{Main}-Klasse, in welcher alle Funktionen eines \toya Programms liegen. Anschließend durchläuft das Backend den abstrakten Syntaxbaum. Jede Klasse des abstrakten Syntaxbaums besitzt eine Generierungs-Funktion, die den richtigen Bytecode für den Knoten des abstrakten Syntaxbaumes erzeugt. Sobald der abstrakte Syntaxbaum fertig durchlaufen ist, erzeugt \toya eine ausführbare class-Datei. Ist ein \toya Programm über mehrere Dateien verteilt, fasst der Compiler das Ergebnis in einer class-Datei zusammen.